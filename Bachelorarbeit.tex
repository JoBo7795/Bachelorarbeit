\documentclass[a4paper,12pt]{scrartcl}
 
\usepackage[utf8]{inputenc} 
\usepackage[ngerman]{babel}
\usepackage[T1]{fontenc}
\usepackage{amsmath}
\usepackage{hyperref}
\usepackage{tikz}
\usetikzlibrary{automata,positioning}
\usepackage{pdfpages}

\bibliographystyle{unsrt}

\hypersetup{
    colorlinks,
    citecolor=black,
    filecolor=black,
    linkcolor=blue,
    urlcolor=black
}



\title{}


\author{Johannes Bohlig}

\date{\today}


\begin{document}

\tableofcontents

\pagebreak

\section{Einleitung}

Als Anbieter von Cloud-Services hat die Firma Appointrix den Anspruch, möglichst geringe und kurze Ausfallzeiten mit ihren Services zu erreichen. Um eine maximal lange, störungsfreie Servicelaufzeit zu erreichen, ist es notwendig jederzeit den Servicestatus einsehen und mögliches Fehlverhalten frühzeitig erkennen zu können. 
Da für eine dauerhafte Kontrolle eines Services ein oder sogar mehrere Mitarbeiter benötigt werden, welche eine eintönige Kontrollaufgabe übernehmen müssten, ist es sinnvoll möglichst viele Teile der Kontrolle zu automatisieren. Diese Automatisierung bringt einerseits den Vorteil der Kosteneinsparung, da keine Mitarbeiter für diese Aufgabe benötigt werden und andererseits einen Geschwindigkeitsvorteil durch die wesentlich geringere Reaktionszeit, die durch die Geschwindigkeit von Computern gegenüber dem Menschen einhergeht.\\
Hierbei sollen vor allem Engpässe bei Ressourcen ausfindig gemacht werden, sowie Anomalien, also Fehlverhalten, in einzelnen Komponenten der Infrastruktur gefunden und behoben werden, bestenfalls noch bevor sich größere Auswirkungen auf die restlichen Komponenten ergeben.\\
Um eine automatisierte Erkennung zu ermöglichen, werden Daten sog. Metriken benötigt, die eine Entscheidung über das vorliegende Verhalten treffen lassen. Metriken müssen erhoben und ausgewertet werden, um eine Aktion aus ihnen schließen zu können, welche die vorliegende Anomalie oder den vorliegenden Engpass beheben kann.\\
Die erhobenen Metriken müssen einerseits für Menschen lesbar sein, um aktuelle Zustände widerspiegeln und entsprechend darauf reagieren zu können, andererseits ebenso für Computer auswertbar sein, um die Automatisierung durch diese zu ermöglichen.
Kubernetes bietet die Möglichkeit, mittels des sog. Kubelet und des cAdvisor Metriken zu erheben, welche den Zustand des Clusters darstellen. \\
Neben automatisierten Aktionen auf das Cluster, ist es auch sinnvoll entsprechende Stakeholder über das Fehlverhalten in Kenntnis zu setzen und diese zu benachrichtigen, um ihnen die Möglichkeit zu geben dem Verhalten auf den Grund zu gehen.\\
Diese Arbeit setzt sich das Ziel die Durchführbarkeit der automatisierten Anomalie- und Engpasserkennung nachzuweisen und die erste Implementierung innerhalb eines schon bestehenden Kubernetes-Clusters. 
Im Rahmen dieser Arbeit werden die passenden Komponenten gewählt, die zur Umsetzung der Anforderungen benötigt werden, die Infrastruktur geplant, erstellt und die korrekte Funktion evaluiert.\\
Des Weiteren wird die Relevanz verschiedener erhobener Metriken in Bezug auf ihre Verwendbarkeit beim automatisierten Detektieren von Anomalien und Engpässen dargestellt und geklärt.\\
Es werden die weit verbreiteten Tools Prometheus und Grafana verwendet und durch Eigenentwicklungen ergänzt und so eine Infrastruktur geschaffen, welche die Anforderungen erfüllen kann.

\pagebreak

\subsection{Struktur}

Die nachfolgende Arbeit ist wie folgt strukturiert:\\

\begin{description}

\item [Kapitel 2] befasst sich mit den Grundlagen die für das weitere Verständnis der Arbeit erforderlich sind.
\item [Kapitel 3] stellt den aktuellen Stand der Technik dar.
\item [Kapitel 4] erläutert Tools und Vorgehensweise bei der Datenakquise
\item [Kapitel 5] erläutert das Vorgen beim Auswerten der Metriken
\item [Kapitel 6] befasst sich mit den Automatisierten Aktionen, die aus den ausgewerteten Metriken geschlossen werden können
\item [Kapitel 7] zeigt, wie aufgestellte Regeln und Komponenten evaluiert werden
\item [Kapitel 8] diskutiert die Ergebnisse dieser Arbeit
\item [Kapitel 9] enthält das Fazit, sowie einen Ausblick in die Zukunft des Projekts

\end{description}

\section{Grundlagen}
\subsection{Kubernetes}

Kubernetes ist eine Open-Source Software und API zur Orchestrierung und Deployment von containerisierten Anwendungen. Seit seiner Einführung 2014 hat Kubernetes ein starkes Wachstum erlebt und ist zum Quasistandard bei der Entwicklung von Cloud-Native Applikationen geworden. Kubernetes ist eine mittlerweile bewährte Infrastruktur und bietet Software die nötig ist um zuverlässige und skalierbare verteile Systeme zu entwickeln. \cite{Burns.2019}

\subsubsection{Grundlegender Aufbau}



\subsubsection{Relevante Komponenten}

Kubernetes bietet eine Vielzahl von Komponenten für unterschiedlichste Aufgaben. Die für dieses Projekt relevanten Komponenten werden hier erklärt:

\begin{description}



\item [kubelet]:
\item [Deployment]:
\item [Service]:
\item [cAdvisor]:\\
Da Container von sich aus keine Informationen zu ihrem Ressourcenstatus nach außen preisgeben oder exportieren, bedarf es eines Hilfsmittels, das genau dies macht.
cAdvisor(Container Advisor) ist ein Daemon, der Ressourcen-Informationen aus Containern sammelt, verarbeitet und exportiert.\cite{.20200704T23:29:24.000Z}
\item []:
\item []:

\end{description}

\subsection{Cloud-Native}
\subsection{Metriken}
\subsection{Prometheus}
\subsubsection{Scraping}
\subsubsection{Abfragen}
\subsubsection{Alerting}

\section{Stand der Technik}

\section{Datenaggregation}
\subsection{Toolauswahl}
\subsubsection{Metriken}
\subsubsection{Visualisierung}
\subsection{Datenquellen}
\subsection{'USE'-Methode}

\section{Auswerten der Metriken}
\subsection{Klassifizierung}
\subsection{Logische Auswertung}
\subsection{Graphische Aufbereitung}

\section{Automatisierte Aktionen}
\subsection{Aktionen}
\subsubsection{Skalieren}
\subsubsection{Anomalie-Detection}
\subsection{Komponenten und Architektur}
\subsubsection{Prometheus}
\subsubsection{Alertmanager}
\subsubsection{Alert-Action-Manager}
\subsection{Regeln}
\subsubsection{Metriken}
\subsubsection{Grenzwerte}

\section{Evaluation}
\subsection{Regeln}
\subsection{Grenzwerte}
\subsection{Komponenten und Architektur}

\section{Diskussion}
\section{Fazit und Ausblick}

\newpage
\bibliography{Literatur}

\end{document}