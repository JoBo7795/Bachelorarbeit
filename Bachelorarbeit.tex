\documentclass[a4paper,12pt]{scrartcl}
 
\usepackage[utf8]{inputenc} 
\usepackage[ngerman]{babel}
\usepackage[T1]{fontenc}
\usepackage{amsmath}
\usepackage{hyperref}
\usepackage{tikz}
\usetikzlibrary{automata,positioning}
\usepackage{pdfpages}



\hypersetup{
    colorlinks,
    citecolor=black,
    filecolor=black,
    linkcolor=blue,
    urlcolor=black
}



\title{}


\author{Johannes Bohlig}

\date{\today}


\begin{document}

\tableofcontents

\pagebreak

\section{Einleitung}
\subsection{Motivation}
\subsection{Problemstellung}
\subsection{Ziel}
\subsection{Ansatz}
\subsection{Struktur}

\section{Grundlagen}
\subsection{Kubernetes}
\subsubsection{Funktionsweise}
\subsubsection{Komponenten}
\subsection{Metriken}
\subsection{Prometheus}
\subsubsection{Scraping}
\subsubsection{Abfragen}
\subsubsection{Alerting}
\subsection{Grafana}

\section{Stand der Technik}

\section{Datenaggregation}
\subsection{Datenquellen}
\subsection{'USE'-Methode}

\section{Auswerten der Metriken}
\subsection{Logische Auswertung}
\subsection{Graphische Aufbereitung}

\section{Automatisierte Aktionen}
\subsection{Aktionen}
\subsubsection{Skalieren}
\subsubsection{Anomalie-Detection}
\subsection{Komponenten und Architektur}
\subsubsection{Prometheus}
\subsubsection{Alertmanager}
\subsubsection{Alert-Action-Manager}
\subsection{Regeln}
\subsubsection{Metriken}
\subsubsection{Grenzwerte}

\section{Evaluation}
\subsection{Regeln}
\subsection{Grenzwerte}
\subsection{Komponenten und Architektur}

\section{Diskussion}

\section{Fazit}

\section{Ausblick}


\end{document}